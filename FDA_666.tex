\documentclass[12pt,letterpaper]{article}
\usepackage{epsfig}
\usepackage{amsmath}
\usepackage{amsfonts}
\usepackage{amssymb}
\usepackage{amstext}
\setlength{\topmargin}{-.4in}
\setlength{\textheight}{9in}
\setlength{\oddsidemargin}{.125in}
\setlength{\textwidth}{6.25in}
\usepackage{graphicx}
\usepackage{color}
\newcommand{\blue}{\color{blue}}
\newcommand{\red}{\color{red}}

\begin{document}

\title{\textbf{\Large Financial Data Analysis}}
\author{Qing Lou, Qianqian Huang, Luyao Gao, Chunlin Liu}
\date{}
\maketiitle 


\begin{abstract}
This is abstract.
\end{abstract}


\section{Introduction}

\noindent 
Real estate economics is gradually becoming an irreplaceable facet of social topic in metropolis. 
The captivate of the preemption of real estate is a double-edged sword that not only polarizes the rich and the poor, but also contains the incentives of the middle-class people of having house possession.
One preponderant element of this trading is the house price, while the much latent compounds lying behind it could be the lending rate of this peticular kind of housing loan.
We are considering the lending rate in that the central government execute the management control system of the monetory market though various policy tool, the lending rate occupies an important position.
The common sense makes it possible that the influence of the leading rate on the housing price is undeniable, but we still curious about the range of this effect.
In our ayalysis of the relationship between the housing price and lending rate, the Autoregressive (AR) model fits the data by the order 12. 
Although this recognizable 12 provides no evidence to quarrel with, the Autoregressive Integrated Moving Average (ARIMA) Model illustrates the connection by more explanatory variables,especially the one from the error term in AR model.
Also, the General Autoregressive conditional heteroskedasticity (GARCH) model is used, but it does not yield the significance contribution on data fitting.
+conlcusion

\section{Literature Review}

\section{Setting / Model}

\section{Data}
         The data we choose are two typical financial products: one is the Goverment bond and 
         the other bond is Corperation stock. One of the ultimate goals of our project is to 
         estimated the ARMA model. Because the characteristics of the two financial products 
         are greatly different to some extent, we want to figure out whether we will get two 
         different ARMA model with different lags.

\section{Estimation Method}

\section{Empirical Results}

\section{Discussion}

\section{Conclusion}




















\end{document}
